\documentclass{article}

\usepackage{graphicx}
\usepackage{latexsym}
\usepackage{textcomp}
\usepackage{longtable}
\usepackage{multirow, booktabs}
\usepackage{amsfonts,amsmath,amssymb}
\usepackage{natbib}
\usepackage{url}
\usepackage{hyperref}
\usepackage{amsmath}
\usepackage{mathtools}
\usepackage{chemarrow}
\usepackage{subcaption}
\usepackage{amsmath}
\usepackage{mathtools}
\usepackage{rotating}
\usepackage{multirow}
\usepackage{tikz}
\usepackage[margin=.75in]{geometry}

\renewcommand{\baselinestretch}{1.5}
\newcommand{\bl}[1]{{\mathbf #1}}
\newcommand{\bs}[1]{{\boldsymbol #1}}
\newcommand{\tr}{\text{tr}}
\newcommand{\etr}{\text{etr}}
\newcommand{\Exp}[1]{{\text{E}}[ \ensuremath{ #1 } ]  }

\begin{document}

\title{Shared Subspace Model}


\date{\today}


\title{Shared Subspace Covariance Model}
\author{Alexander Franks and Peter Hoff}
\date{\today}
\maketitle 



\begin{abstract}

% From Hoff Although the covariance matrices corresponding to different
% populations are unlikely to be exactly equal they can still exhibit a
% high degree of similarity. For example, some pairs of variables may be
% positively correlated across most groups, whereas the correlation
% between other pairs may be consistently negative. In such cases much
% of the similarity across covari- ance matrices can be described by
% similarities in their principal axes, which are the axes that are
% defined by the eigenvectors of the covariance matrices. Estimating the
% degree of across-popu- lation eigenvector heterogeneity can be helpful
% for a variety of estimation tasks.



  We develop model-based methods for evaluating similarities and
  differences among several $p\times p$ covariance matrices in the
  $p > n$ setting.  This is done by assuming a spiked covariance model
  for each group, and sharing information about the subspace spanned
  by the group-level eigenvectors.  We propose an EM algorithm for
  estimating the shared subspace and the covariance matrix of the data
  projected onto this subspace.  We also provide a Gibbs sampling
  algorithm for estimating the posterior uncertainty of group-level
  eigenvectors and eigenvalues.  Finally, we demonstrate the utility
  of our method in analyses exploring covariability in biological data.

\end{abstract}

% \section{NOTES}

% \begin{itemize}
% \item For MSCov to work in no-pooeld case need $R < S$ (doens't work if $S=P$).
% \item Explore the effect on estimation when R=2 two eigenvalues converge/diverge.
% \item TODO: Experiments: 
% \begin{itemize}
% \item S=20, R=2, data generated under no-pooled model.
% \item Fit with R=2 to R=20, Fix S=20
% \item Fit with R=2, S=5 to 20
% \item Fix $1/(1-Y^2)$ bug for Y=1
% \item debeta_un = 0 for large sample sizes (underflows)
% \item Check that CP models yield same results using fitBayesianSpike and fit-subspace.
% \end{itemize}
% \end{itemize}



\section{Introduction}

Introduction to covariance estimation and spiked covariance model.

Suppose $\bl S$ has a possibly degenerate Wishart$(\Sigma,n$) 
distribution with density 
given by 
\begin{equation} 
p(\bl S | \Sigma, n) \propto_{\Sigma} l(\Sigma:\bl S) =  |\Sigma|^{-n/2} \text{etr}( - \Sigma^{-1} \bl S/2 ) ,  
\label{eqn:lik}
\end{equation}
where $\Sigma \in \mathcal S_p^+$ and $n$ may be less than $p$. 

% However, when the population size is large and comparable with the
% sample size, as is in many contemporary data, it is known that the
% sample covariance matrix is no longer a good approximation to the
% covari- ance matrix. Baik
 
Such a likelihood results from $\bl S$ being, for example, a residual 
sum of squares matrix from a multivariate regression analysis. In  this case,
$n$ is the number of independent observations minus the 
rank of the design matrix. 
The spiked principle 
components model (spiked PCA) introduced by \citet{Johnstone2001} assumes that 
\begin{equation} 
\Sigma = \sigma^2 (  \bl U  \Lambda  \bl U^T  + \bl I )
\label{eqn:spiked}
\end{equation}

where $\Lambda$ is an $r\times r$ diagonal matrix and 
$\bl U \in \mathcal V_{r,p}$, with 
%$r< ( n \wedge p)$. 
$r<p$. Thus, the spiked covariance model is a low rank factor model
with homoscedastic error \citep{}.  Such a formulation is appealing because it explicitly
partitions the covariance into a tractable low rank ``signal'' and homscedastic ``noise''.  

For such a covariance matrix, we have
$\Sigma^{-1} = (\bl U \Lambda \bl U^T + \bl I )^{-1} = \bl I - \bl U
\Omega \bl U^T $
where $\Omega = \text{diag}(\omega_1,\ldots, \omega_r)$ with
$\omega_s = \lambda_j/(\lambda_s+1)$.  The likelihood (\ref{eqn:lik})
in terms of $(\sigma^2,\bl U, \Omega)$ is then
\begin{equation}
L(\sigma^2,\bl U , \Omega : \bl Y)  = 
(\sigma^2)^{-np/2} \etr(-\bl S/[2\sigma^2]) \times 
   \etr( \Omega \bl U^T [\bl S/(2\sigma^2)] \bl U ) \times  
 \left\{ \prod_{s=1}^r (1-\omega_s)^{n/2} \right \} .  
\label{eqn:rplik}
\end{equation}

Classical results for parametric models 
(e.g., \citet{schwartz_1965})  
imply that 
asymptotically in $n$ for fixed $p$, a Bayes estimator
will be consistent for a spiked  population covariance 
as long as the assumed number of spikes is greater than or equal to 
the true number. 
However, the situation is more difficult 
when both $n$ and $p$ are large.  
%Since a Bayes estimator for an invariant loss function 
%is orthogonally equivariant under this prior, 
%it can be expressed as 
%$\hat \Sigma(\bl S) = \bl E \bl D(\bl L) \bl E^T$, 
%where $\bl E$ and $\bl L$ are the matrices of eigenvectors and 
%eigenvalues of $\bl S$, and $\bl D$ denotes a diagonal matrix 
%that is a function of $\bl L$. 
%As a result 
Under the spiked covariance model, 
results of 
\citet{Baik2006, Paul2007} indicate that if 
$p/n \rightarrow \gamma >0$ as $n\rightarrow \infty$,   
the top $r$ eigenvalues of $\bl S/(n\sigma^2)$ will converge 
to upwardly  biased versions of $(\lambda_1+1,\ldots, \lambda_r+1)$, 
if each $\lambda_s $ is greater than $\sqrt{\gamma}$.   
This has led several authors to suggest estimating 
$\Sigma$ via shrinkage of the eigenvalues of the sample covariance 
matrix. In particular, 
in the setting that $\sigma^2$ is known, 
\citet{Donoho2013} propose  
setting all sample eigenvalues that fall below 
the value $(1+\sqrt{\gamma})^2$ to be one, and shrinking the larger 
eigenvalues in a way that depends on the particular 
loss function being used. 
These shrinkage functions are shown to be asymptotically 
optimal in the $p/n\rightarrow \gamma$ setting. 

As demonstrated by Gavish et al \citep{Gavish2014}, even when the
true rank is large, if blah it is still asymptotically preferable to fit a lower rank
blah .  it is always better than hard thresholding at any other value, and is always better than ideal truncated singular value decomposition (TSVD)

Bayes estimators under the prior distribution described above 
can also be viewed as eigenvalue shrinkage estimators: 
Since a Bayes estimator for an invariant loss function 
is orthogonally equivariant under this prior, 
it can be expressed as 
$\hat \Sigma(\bl S) = \bl E  \hat \Sigma (\bl L) \bl E^T$, 
where $\bl E$ and $\bl L$ are the matrices of eigenvectors and 
eigenvalues of $\bl S$. Furthermore, it is straightforward to 
show that $\hat \Sigma (\bl L)$ must be a diagonal matrix, 
and so the Bayes estimator is obtained by applying the 
shrinkage function $\hat\Sigma$ to the sample eigenvalues $\bl L$. 
Understanding of this shrinkage function could in principle 
be pursued by investigation of the posterior distribution of the 
eigenvalue parameters $\Lambda$, or equivalently their  
transformed values $\Omega$.   

However, in many modern applications, it is not enough to
independently estimate an individual covariance matrix.  Rather, a
comparative analysis of covariance patterns across multiple groups of
observations would be most informative.  For example, in case
control-studies, mean level effects may be small relative to subject
variability.  Differences in the covariance
matrices may lead to better insights about the differences between groups.
Alternatively, when mean level differences are detectable, better
estimates of the covariability of features across groups may lead to improved
understanding of the mechanisms underlying these apparent mean level
differences.

With this situation in mind, Flury (1984) developed estimation and
testing pro-cedures for the ‘common principal components’ model, in
which a set of covariance matrices
\citep{Flury1987} and \citep{Schott1991, Schott1999} considered
cases in which only certain columns or subspaces of U are shared across populations, and \citep{Boik2002} described a very general model in which eigenspaces can be shared between all or some of the populations.

Continuously parameterized covariance matrices. \citep{Hoff2011}, \citep{Chiu1996}, \citep{Yin2010}

In such models, it is no longer reasonable to assume orthogonal equivariance .
To address these issues, \citet{Hoff2009} introduced a model for
eigenvector shrinkage based on the matrix Bingham distribution.

Partial least squares discriminant analysis refs?
\citep{Westerhuis2008, Szymanska2012, Eriksson2001, Barker2003, Saebo2008}

Recently there has been significant interest gaussian graphical
models, .   In gaussian graphical models, zeros in the
precision matrix correspond to conditional independences between pairs
of features given the remaining features \citep{}.  \citet{Witten2014} developed a
method for pooled estimation inverse covariance matrices across
multiple groups.  Identifying a shared pattern of zero's in the inverse covariance
matrices for multiple classes of observations. 

In the following section, we
propose a model for estimating and exploring differences between
covariance matrices from multiple groups of related data.

Below we describe our approach for comparaing covariance matrices on
a common low dimensional subspace.

\section{The Shared Subspace Spiked Covariance Model}
\label{sec:shared}

Assume we have $k=1, ..., K$ groups of matrices $\bl S_k$ each of
which follow a (degenerate) Wishart distribution, as in Equation
\ref{eqn:lik}.  Each group consists of of $n_k$ measurements of $p$
features, typically with $n_k << p$.  We pool information about the
covariance matrices by assuming that most of the relevant variation
for each group shares a low rank subspace spanned by
the columns of a matrix $V$.  That is,

\begin{equation}
\Sigma_k = V\psi_kV^T + \sigma^2_kI
\label{eqn:sspsi}
\end{equation}

Here, $V \in \mathcal{V}_{s,p}$ and $p \geq s$.  $V$ is only
identifiable up to right rotations.  Consequently,
$VV^T \in \mathcal{G}_{s, p}$, which defines the plane of variation
shared by all groups, is identifiable.  $\psi_k$ is a rank-$r$
covariance matrix on the $s$-dimensional subspace.  Such a model is
reasonable for many applications involving groups of highly structure
data on the same set of $p$ features.  For instance, in biological
applications, only certain subsets of molecules can physically
interact which constrains the space of possible correlation patterns.
Our model exploits this notion, by estimating differences between
groups on a common lower dimensional subpace.

\begin{figure}[t]
    \centering
    \begin{subfigure}[b]{0.25\textwidth}
        \includegraphics[width=\textwidth]{Figs/3dplot}
        \caption{Ratio}
        \label{fig:dmelanRatio}
    \end{subfigure}
    \qquad     \qquad     \qquad 
 %add desired spacing between images, e. g. ~, \quad, \qquad, \hfill etc. 
      %(or a blank line to force the subfigure onto a new line)
    \begin{subfigure}[b]{0.25\textwidth}r
        \includegraphics[width=\textwidth]{Figs/2d-scatter}
        \caption{Posterior}
        \label{fig:dmelanPosterior}
    \end{subfigure}
    \caption{ }
\end{figure}


\section{Inference}

In this section we outline Bayesian inference algorithms including
both an optimization algorithm and an MCMC algorithm for estimating
covariance parameters and their associated uncertainty.  MCMC
inference is particularly appealing because of the ease with which we
can quantify the uncertainty associated with each estimate.
Uncertainty quantification is particularly important in the $n << p$
setting where seemingly large differences in the point estimates of
the low rank covariance matrices across groups may not actually
reflect significant differences.  Although, uncertainty quantification
for all parameters is desirable, for the size of the problems we are
generally interested in, we found MCMC inference for the subspace
$VV^T$ to be intractable.  Our optimziation based approached is
signifcantly faster and works well for large large dimensional
datasets.  As such, we start, in Section \ref{sec:em}, by outlining an
expectation-maximization algorithm for estimating the shared subspace
and derving point estimates of the projected covariance $\psi_k$. In
Section \ref{sec:bayes} we describe a Gibbs sampler for posterior
inference of $\psi_k$ given the common subspace.  These two algorithms
can be used in conjuction to first identify and evaluate a shared
subspace and then to quantify the uncertainty associated with the
projected covariance matrices on this space.

\subsection{An Expectation-Maximization Algorithm}
\label{sec:em}

In this section we describe a computationally efficient algorithm for
estimating the shared subspace, $VV^T$, projected covariance matrix
$\psi_k$ and noise parameter $\sigma_k^2$.  First, note that the full
likelihood for the shared subspace model can be written as

\begin{align}
p(S_1, ... S_k | \Sigma_k,n_k) &\propto \prod_{k=1}^K |\Sigma_k|^{-n_k/2}\etr(-\Sigma_k^{-1}\mathbf{S}_k/2)  \\
&\propto \prod_{k=1}^K  |\Sigma_k|^{-n_k/2}\etr(-(\sigma_k^2(V\psi_kV^T +
  I))^{-1}\mathbf{S}_k/2) \\
&\propto \prod_{k=1}^K  |\Sigma_k|^{-n_k/2}\etr(-\left[V(\psi_k +
  I)^{-1}/\sigma_k^2 V^T + (I-VV^T)/\sigma^2_k\right]\mathbf{S}_k/2)
  \\
&\propto \prod_{k=1}^K  (\sigma_k^2)^{-n_k(p-s)/2}|M_k|^{-n_k/2}\etr(-\left[VM_k^{-1}V^T + \frac{1}{\sigma^2_k} (I-VV^T)\right]\mathbf{S}_k/2) 
\end{align}

Where we define $M_k^{-1} = (\psi_k + I) ^{-1}/\sigma_k^2$.  The log-likelihood in
$V$ (up to additive constant) is

\begin{align}
l(V) &= \sum_k tr\left(-VM_k^{-1}V^T +
       VV^T/\sigma^2_k\right)\mathbf{S}_k/2)\\
&= \sum_k \frac{1}{2}tr\left(-M_k^{-1}V^T \mathbf{S}_kV\right) + \frac{1}{2\sigma_k^2}tr\left(VV^T \mathbf{S}_k\right)
\end{align}

% \begin{align}
% p(V|...) &\propto \etr(\sum_{k=1}^K[O_k\Omega_kO^T_k]V^TS_k/(2\sigma^2_k)]V)\\
% \end{align}

We propose an expectation-maximization (EM)
algorithm, to estimate the relevant quantities.  In particular we
maximize the marginal posterior distribution of $V$, treating 
$M_k^{-1}$ and $\frac{1}{\sigma_k^2}$ as the ``missing'' parameters to
be imputed.  We assume independent Jeffreys' prior distributions for
both $\frac{1}{\sigma_k^2}$ and $M_k^{-1}$ and a uniform prior for $V
\in \mathcal{V}_{s, p}$.  In particular we let
$p(\frac{1}{\sigma_k^2}) \propto \sigma_k^2$ and
$p(M_k^{-1}) \propto |M_k^{-1}|^{-(s+1)/2}$.  From the likelihood it
can easily be shown that the conditional posterior for $M_k^{-1}$ is

$$p(M_k^{-1} | V) \propto |M_k^{-1}|^{(n_k - s -1)/2}\etr(-1/2(M_k^{-1}V^T\mathbf{S}_kV)) $$

\noindent which is a Wishart($V^T\mathbf{S}_kV$, $n_k$).  The
conditional posterior distribution of $\frac{1}{\sigma_k^2}$ is simply

$$p\left(\frac{1}{\sigma^2} | V\right) \propto \left(\frac{1}{\sigma_k^2}\right)^{n_k(p-s)/2-1}\etr\left(-\frac{1}{2\sigma^2_k} (I-VV^T)\mathbf{S}_k\right)  $$

\noindent which is a Gamma($n_k(p-r)/2$,
$\frac{1}{2}tr((I-VV^T)\mathbf{S}_k)$).  Thus, for each iteration,
$t$, of the EM algorithm:

\begin{enumerate}
\item For each $k$, compute relevant conditional expectations:
\begin{enumerate}
\item $E[M_k^{-1} | V_{(t-1)}] = n_k(V_{(t-1)}^T \mathbf{S}_kV_{(t-1)})^{-1}$
\item $E[\frac{1}{2\sigma_k^2}|V_{(t-1)}] = \frac{n_k(p-r)}{tr((I-V_{(t-1)}V_{(t-1)}^T)S_k)}$
\end{enumerate}
\item Compute $V_{t}$ = $\underset{V}{argmax}  \sum_k tr\left(-VE[M^{-1}|V_{(t-1)}]^T +
       E[\frac{1}{2\sigma_k^2}|V_{(t-1)}]VV^T\right)\mathbf{S}_k/2)$ 
\end{enumerate}

For the ``M-step'', we use a numerical optimization algorithm based on
the Cayley transform to preserve the orthogonality constraints in $V$
\citep{Wen2013}.  Importantly, the complexity of this algorithm is
dominated by dimension of the shared subspace, $s$, not the number of
features $p$.  Thus, our approach is computationally efficient for
relatively small values of $s$, even when $p$ is large.



\paragraph{Evaluating a Shared Subspace}

The above EM algorithm provides an estimate of a shared subspace, but
in practice there are no guarantees that the inferred subspace is
equally appropriate for all groups.  Moreover, the shared subspace can
be chosen in other ways, including more ad hoc methods
involving the singular value dcomposition or simply by incorporating
subject matter considerations.  Regardless of how a particular
subspace is identified, it is usually important to quantify how useful
this subspace is for explaining the variability in each group.  In
particular, if $\hat{V}$ is a basis for a subspace shared across
groups, then for each $k$, most of the variation in $\Sigma_k$ should
lie on the subspace spanned by $\hat{V}$.  To quantify the extent to
which this is true for different groups, we propose a simple estimator
for the proportion of ``signal'' variance that lies on any given
subspace.  Specifically, we apply the following estimator for ratio of
the sum of the first $s$ eigenvalues of
$\hat{V}^T\bl \Sigma_k \hat{V}$ to the sum of the first $s$
eigenvalues of $\Sigma_k$:

\begin{equation}
 \frac{||Y_k^T\hat{V}||_F}{\underset{\widetilde{V} \in \mathcal{V}_{p, s}}{sup}
  ||Y_k^T\widetilde{V}||_F - \hat{\sigma}_k^2ps}
\label{eqn:ratio}
\end{equation}

where $||.||_F$ is the Frobenius norm. If the inferred subspace is the
true subspace, e.g. $\hat{V}\hat{V}^T = VV^T$, then this estimator
will asymptotically approach one.  To see this, note that
$\underset{V \in \mathcal{V}_{p, s}}{sup}\frac{||Y_k^TV||_F}{n_k}$ is
equivalent to the sum of the first $s$ eigenvalues of the sample
covariance matrix $\frac{\bl S_k}{n_k}$.  With $\hat{\lambda}^{(k)}_i$
the $i$-th eigenvalue of $\frac{\bl S_k}{n_k}$, \citet{Baik2006}
demonstrated that asymptotically as $p, n_k \rightarrow \infty$ with
$\frac{p}{n_k} = \gamma_k$ fixed

\begin{eqnarray}
\hat{\lambda}^{(k)}_i &\rightarrow& \lambda^{(k)}_i\left(1 +
                                    \frac{\sigma_k^2\gamma}{\lambda^{(k)}_i
                                    - 1}\right)\\
& \approx& \lambda^{(k)} + \sigma^2_k\frac{p}{n_k}
\end{eqnarray}

Thus,
$\sum_i^s \hat{\lambda}^{(k)}_i \approx \sum_i^s \lambda^{(k)}_i +
\sigma_k^2ps$.
If we assume that $s$ is fixed in this asymptotic regime, then
$||Y_k^TV||_F/n_k$ is a consistent estimator for
$\sum_i^s \lambda^{(k)}_i$, and thus the above estimator will be close
to one when $\hat{V}\hat{V}^T = VV^T$ and smaller if not.  The metric provides a
useful indicator for which groups can be reasonably compared on a
given subspace and which groups cannot.  In Section \ref{sec:simulation} we plot
these estimates for different simulated datasets.  In Section \ref{sec:bio}
we plot the estimates for inferred subspaces for real biological data.

\subsection{Uncertainty Quantification}
\label{sec:bayes}

Although the EM algorithm presented in the previous section yields a
fast way to derive point estimates for $VV^T$ and $\psi_k$, it does
not lead to natural uncertainty bounds for these estimates. Full
uncertainty quantification fo $VV^T$ is desirable but we found this
challenging for the dimensions of the problems we would like to
consider.  As such, in this section, we assume that the subspace
$VV^T$ is fixed and known and focus on uncertainty quantificatoin for
$\psi_k$ only.  Further, by focusing on group differences on a fixed
subspace, it is simpler to interpret differences across covariance
matrices.  The shared subspace can be an estimated using the EM algorithm
presented in Section \ref{sec:em} or chosen based on subejct matter
considerations.  Note that when the subspace is known, the posterior
distribution of $\Sigma_k$ is conditionally independent from the other
groups, i.e.
$P(\Sigma_{k} | V, \bl S_k, \Sigma_{-k}) = P(\Sigma_{k} | V, \bl
S_k)$,
so that we can independently estimate the conditional posterior
distribution for each group.

There are many different prior specifications for $\psi_k$ and
$\sigma_k^2$ that we could consider for estimating posterior
uncertainty.  One simple approach would be to use the same prior
specification used in Section \ref{sec:em}.  That is, we can use
Jeffreys' prior for $\frac{1}{\sigma_k^2}$ and
$M_k^{-1} = (\psi_k + I) ^{-1}/\sigma_k^2$.  In fact, it is easy to
show that the Gamma distribution for $\frac{1}{\sigma_k^2}$ and the
Wishart distribution for $M_k^{-1}$ are conjugate priors for the
respective quantities.  From posterior samples of $M_k$ and
$\sigma_k^2$ we can easily recover samples of $\psi_k$.
Unfortunately, if $M_k^{-1}$ has a Wishart distribution, the implied
support of $\psi_k$ is not constrained to the cone of positive
semi-definite matrices.  

Direct prior specifications for $\psi_k$ are more appealing, but many
obvious choices, like the inverse-Wishart prior distribution for
$\psi_k$ do not lead to simple posterior distributions.  As such,
building on recent interest in spiked covariance models
\citep{Hoff2009, Donoho2013, Paul2007} we focus on tractable models
based on the eigendecomposition of $\psi_k$.  Such models are
especially appealing because it is simple to summarize and compare
$\psi_k$ across groups in terms their eigenstructure (See Section
\ref{sec:simulation} and \ref{sec:bio}).

By modeling the eigenstructure, we can now view each covariance
$\Sigma_k$ as a spiked covariance matrix.  Equation \ref{eqn:sspsi}, written as a function $V$, becomes

\begin{equation}
\Sigma_k = VO_k\Lambda_KO_k^TV^T + \sigma^2_kI
\label{eqn:ss}
\end{equation}

Here, we make one further generalization to allow
$\psi_k = O_k\Lambda_KO_k^T$ be a rank $r \leq s$ dimensional covariance
matrix on the $s$-dimensional subspace.  Thus, $\Lambda_k$ is an $r \times
r$ diagonal matrix of eigenvalues, and $O_k \in
\mathcal{V}_{s,r}$ is the matrix of eigenvectors of
$\psi_k$.  For any individual group, this corresponds to the original spiked
covariance model (Equation \ref{eqn:spiked}) with $U_k = VO_k \in
\mathcal{V}_{p, r}$.

The likelihood for $\Sigma_k$ given the sufficient statistic $\mathbf{S}_k = Y_kY_k^T$ is simply

\begin{equation}
p(S | \Sigma_k,n_k) \propto |\Sigma_k|^{-n/2}\etr(-\Sigma_k^{-1}\mathbf{S}_k/2)
\end{equation}

First note that by the Woodbury matrix identity we can show that 
\begin{align}
\Sigma^{-1}_k &=  (\sigma_k^2(U_k\Lambda_kU_k^T+I))^{-1}\\
&= \frac{1}{\sigma_k^2}(U_k\Lambda_kU_k^T+I)^{-1}\\
&= \frac{1}{\sigma_k^2}(I-U_k\Omega_kU_k^T)\\
\end{align}

Where the diagonal matrix $\Omega = \Lambda(I+\Lambda)^{-1}$, e.g. $\omega_i = \frac{\lambda_i}{\lambda_{i}+1}$.  Further, 

\begin{align}
|\Sigma_k| &= (\sigma_k^2)^{p}|U_k\Lambda_kU_k^T+I|\\
&= (\sigma_k^2)^{p}|\Lambda_k+I| \\
&= (\sigma_k^2)^{p}\prod_{i=1}^r(\lambda_i+1)\\
&= (\sigma_k^2)^{p}\prod_{i=1}^r(1-\omega_i)
\end{align}

where the second line is due to Sylvester's determinant theorem.  The
likelihood of $V$, $O_k$, $\Lambda_k$ and $\sigma_k^2$ is available
from Equation\ref{eqn:rplik} by substituting in the appropriate
quantities in for $\Sigma^{-1}_k$ and $|\Sigma_k|$ and replacing $U_k$
with $VO_k$:

\begin{equation}
 L(\sigma_k^2,\bl V , O_k \Omega_k : \bl Y_k) \propto
    (\sigma_k^2)^{-n_kp/2}\etr(-\frac{1}{2\sigma_k^2}\mathbf{S}_k)\left(\prod_{i=1}^r(1-\omega_{ki})
   \right) ^{n_k/2}
   \etr(\frac{1}{2\sigma_k^2}(VO_k\Omega_kO_k^TV^T)\mathbf{S}_k)
\label{eqn:sslik}
\end{equation}

We use conjugate and semiconjugate priors for the parameters $O_k$,
$\sigma^2_k$ and $\Omega_k$ to faciliate inference via a Gibbs
sampling algorithm.  In the absence of real prior information,
invariance considerations suggest the use of priors that lead to
equivariant estimators.  Below we describe our choices for the prior
distributions of each parameter and the resultant conditional posterior
distributions.

\paragraph{Conditional distribution of $\sigma_k^2$:}

From Equation \ref{eqn:sslik} it is clear that the the inverse-gamma
class of prior distributions is conjugate for$\sigma_k^2$.  We chose a
default prior distriubtion for $\sigma^2$ that is equivariant with
respect to scale changes.  Specifically, we use Jeffreys' prior, an
improper prior with density $p(\sigma^2) \propto 1/\sigma^2 $.  Under
this prior, straightforward calculations show that the full
conditional distribution of $\sigma_k^2$ is
inverse-gamma$( n_k p/2 , \tr(\bl S_k(\bl I -\bl U_k\Omega_k\bl
U_k)/2)$, where $U_k = \hat{V}O_k$

\paragraph{Conditional distribution of $O_k$:} Given the likelihood
from Equation \ref{eqn:sslik}, it is easy to show that the class of
von Mises-Fisher-Bingham distributions are conjugate for $O_k$
\citep{Hoff2009, Hoff2012}.  Again, invariance considerations
lead us to use a rotationally invariant uniform probability measure on
$\mathcal V_{s,p}$.  Under this uniform prior, the full conditional
distribution of $\bl O_k$ has a density proportional to the
likelihood, so

\begin{align}
\label{lik_vo}
 p(O_k | \sigma^2_k, U_k, \Omega_k) & \propto \etr(\Omega_kO^T_k\hat{V}^T[S_k/(2\sigma^2_k)]\hat{V}O_k)
\end{align}

\noindent This is a Bingham$(\Omega, \hat{V}^T \bl S_k \hat{V}/(2\sigma^2)$
distribution on $\mathcal V_{s, r}$ \citep{Khatri1977}. A
Gibbs sampler to simulate from this distribution is given in
\citet{Hoff2012}.  

Together, the prior for $\sigma_k^2$ and $O_k$ leads to conditional
(on $V$) Bayes estimators $\hat \Sigma(V^T \bl S V)$ that are
equivariant with respect to scale changes and rotations on the
subspace spanned by $V$, so
that $\hat \Sigma(a \bl W V^T \bl S V \bl W^T) = a \bl W \hat\Sigma(V^T
\bl S V) \bl W$
for all $a>0$ and $\bl W\in \mathcal O_{s}$ (assuming an invariant
loss function). Interestingly, if $\Omega$ were known (which it is
not), then for a given invariant loss function the Bayes estimator
under this prior minimizes the (frequentist) risk among all
equivariant estimators \citep{Eaton1989}.

\paragraph{Conditional distribution for $\Omega$:} Here we specify the conditional
distribution of the diagonal matrix $\Omega_k =
\Lambda_k(I+\Lambda_k)^{-1} = \text{diag}(\omega_{k1}, ... \omega_{kr})$.  We first consider a uniform(0,1) prior distribution for each element of $\Omega$, or
equivalently, an $F_{2,2}$ prior distribution for the elements of
$\Lambda$.  The full conditional distribution of an
element $\omega_i$ of $\Omega$ is proportional to the likelihood
function, so

\begin{align}
p(\omega_{ki}|V, O_k, \bl S_k) \propto_{\omega_{ki}}
  \left(\prod_{i=1}^r(1-\omega_{ki})^{n_k/2}  \right)
  \etr(\frac{1}{2\sigma_k^2}(VO_k\Omega_kO_k^TV^T)\mathbf{S}_k) \\
&  \propto & (1-\omega_{ki})^{n/2} e^{c \omega_s  n/2},    
\label{eqn:wpost}
\end{align}

where $c = \bl u_{ki}^T S_k \bl u_{ki}/(n \sigma^2)$ and $\bl u_{ki}$ is
column $i$ of $U_k = VO_k$.  While not proportional to a density
belonging to a standard class of distributions, we can sample from the
corresponding univariate distribution numerically.  The behavior of
this distribution is straightforward to understand: if $c\leq 1$, then
the the function has a maximum at $\omega_i =0$, and decays
monotonically to zero as $\omega \rightarrow 1$.  If $c>1$ then the
function is uniquely maximized at $(c-1)/c \in (0,1)$.  To see why
this makes sense, note that the likelihood is maximized when the
columns of $\bl U_k$ are equal to the eigenvectors of $\bl S_k$
corresponding to its top $r$ eigenvalues
\citep{Tipping1999}. At this value of $\bl U_k$, $c$ will then
equal one of the top $r$ eigenvalues of $\bl S_k/(n\sigma_k^2)$.  In the
case that $n\gg p$, we expect
$\bl S_k/(n_k\sigma_k^2)\approx \Sigma_k/\sigma_k^2$, the true (scaled)
population covariance, and so we expect $c$ to be near one of the top
$r$ eigenvalues of $\Sigma_0/\sigma^2$, say $\lambda_0+1$.  If indeed
$\Sigma_k$ has $r$ spikes, then $\lambda_k>0$,
$c \approx \lambda_0 +1 > 1$, and so the conditional mode of $w_i$ is
approximately $(c-1)/c = \lambda_i/(\lambda_i+1)$, the correct value.
On the other hand, if we have assumed the existence of a spike when
there is none, then $\lambda_0=0$, $c\approx 1$ and the Bayes estimate
of $w_i$ will be shrunk towards zero, as it should be.



% \paragraph{Conditional distribution of $V$}
% Need to finish / should we include?

% Although we prefer EM... it is also posible to estimate $V$ using
% Bayesian inference.  We assume a uniform prior on $V \in \mathcal{V}_{p, s}$ so that the
% conditional posterior of $V$ is proportional to the likelihood:

% \begin{align}
% p(V|\bl O, \bl S, \bl \sigma^2, \bl \Omega) &\propto \etr(\sum_{k=1}^K[S_k/(2\sigma^2_k)]V[O_k\Omega_kO^T_k]V^T)\\
% \end{align}

% Although close in appearance, this conditional distribution is not a matrix
% Bingham-von Mises-Fisher distribution.  However, the conditional distribution
% of a column of $V$ given all other columns is a vector Bingham-von
% Mises-Fisher distribution.  To see this, let
% $A_k = S_k/(2\sigma^2_k)$ and $B_k = O_k\Omega_kO^T_k$.  We can write
% the conditional distribution of $V$ given $A_k$ and $B_k$ as
% \begin{align}
% P(V|A_k,B_k) &\propto \etr(\sum_{k=1}^KA_kVB_kV^T)\\
% &= \etr(\sum_{k=1}^K\sum_{i,j}^sA_kv_iv_j^Tb_{ijk})\\
% &= \etr(\sum_{i,j}^s \sum_{k=1}^K v_j^T[b_{ijk}A_k]v_i)
% \end{align}

% From here, we specify the conditional distribution of a single column of $V$ given $A_k$, $B_k$ and all other columns of V:

% \begin{align}
% P(v_i|A_k,B_kV_{-i})  &\propto \etr(2(\sum_{j\neq i}^s v_j^T[\sum_{k=1}^K b_{ijk}A_k])v_i+v_i^T[\sum_{k=1}^K b_{ijk}A_k]v_i))\\
% &= \etr(Cv_i+v_i^TMv_i)
% \label{eqn:vbmf}
% \end{align}

% This is a vector Bingham-von Mises-Fisher Distribution, BMF(C, M)
% with $C = 2(\sum_{j\neq i}^s v_j^T[\sum_{k=1}^K b_{ijk}A_k])$ and $M=
% \sum_{k=1}^K b_{ijk}A_k$.  Following \citet{Hoff2012},  we can derive
% a Gibbs sampler for $V$ based on conditional draws of the columns of
% $V$.  


\section{Simulation Study}
\label{sec:simulation}

\subsection{Model Works}
First, we do a single simulation that shows that we can recover the
true parameters using our inferential algorithm when the assumed model
is correct.

\begin{figure}[t]
    \centering
    \begin{subfigure}[b]{0.45\textwidth}
    \includegraphics[width=\textwidth]{Figs/posteriorRegions}
        \caption{Posterior}
        \label{fig:simPosterior}
    \end{subfigure}
    ~ %add desired spacing between images, e. g. ~, \quad, \qquad, \hfill etc. 
      %(or a blank line to force the subfigure onto a new line)
    \begin{subfigure}[b]{0.45\textwidth}
        \includegraphics[width=\textwidth]{Figs/simRatio-05-19}
        \caption{Ratio}
        \label{fig:simRatio}
    \end{subfigure}
    \caption{a) 95\% posterior regions for the ratio of the eigenvalues,
      $\frac{\lambda_1}{\lambda_2}$, of $\psi_k$ and the orientation
      of the principle axis on space spanned by $\hat{V}$ cover the
      truth.  True parameter values on $\hat{V}^T\Sigma_k\hat{V}$ are
      indicated by the crosses.  Since $V$ is only identifiable up to
      rotation we find the Procrustes rotation that maximizes the
      similarity of $\hat{V}$ to the true data generating basis. True
      eigenvalue ratios were 10 (red and black), 3 (green and blue)
      and 1 (light blue).  True orientations were $\pi/4$ (black), $-\pi/4$
      (red and green) and 0 (blue and cyan).  b)  Estimated proportion
    of variance explained by the subspace (Equation \ref{}).  As expected, for each
    group, all of the variance can be explained on the inferred subspace.  }
\end{figure}

In order to quantify the uncertainty associated with the relevant
parameters conditional on a shared subspace, we use the Bayesian
inference algorithm outlined in Section \ref{sec:bayes}. In Figure
\ref{fig:posteriorRegions} we depict the 95\% posterior regions for
statistics based on the eigendecomposition of
$\psi_k$.  Specifically, for the two dimensional
shared subspace model, we plot the eigenvalue ratio
$\frac{\lambda_1}{\lambda_2}$ and the angle of the first eigenvector
on the $\text{arctan}(\frac{O_2}{O_1})$, where
$\lambda_1 > \lambda_2$.  Non-overlapping 95\% posterior regions are
indicative of significant differences in the covariance structure
between groups.

In this example, the posterior regions for all groups cover the true
eigenvalue ratio and orienntation of the primary axis for the matrix
$\hat{V}^T\Sigma_k\hat{V}$.  Note that for smaller eigenvalue ratios,
there is larger uncertainty about the orientation of the primary axis.
When the ratio is one (light blue) there is no information about
orientation of the primary axis since the contours are spherical.

\subsection{Model Misspecification}

We present results of a simulation which demonstrates the
relative performance of shared subspace group covariance estimation
under different data generating models.  In this simulation, we consider three different
spiked covariance data models: 1) ``complete pooling'', in which the
spiked covariance matrices from all groups are identical, e.g.
$\Sigma_1 = \Sigma_2 = ... \Sigma_K = U\Lambda U^T + \sigma^2I$; 2)
the ``shared subspace'' model, outlined in Section \ref{sec:shared};
and 3) ``no pooling'', e.g. the data from each group is generated from
a spiked covariance matrix with each $U_k$ drawn as an independent
uniform sample from the Stiefel manifold $\mathcal{V}_{r, p}$.  For
each of these data generating models, we simulate datasets
with $p=200$ features, rank $r=2$ and eigenvalues
$(\lambda_1, \lambda_2) = (250, 25)$.  For data generated under the
shared subspace model, we additionally assume that the first two
eigenvectors span a common two dimensional subspace, e.g. we fix
$s=2$. 

We evaluate the performance of the three corresponding inferential
models on thirty simulated datasets for each data generating
mechanism.  In practice, the true rank of the covariance matrix is not
known, and thus, prior to fitting each of these inferential models we
must first estimate the rank.  \citet{Gavish2014} provide an
asymptotically optimal (in mean squared error) singular value
threshold for low rank matrix recovery with noisy data.  We use their
rank estimator, which is a function of the median singular value of
the data matrix and $\frac{p}{n_k}$, to independently estimate the
rank, $r$, of the spiked covariance matrices.  For data generated
under the shared subspace model we must also estimate the dimension of
the shared subspace.  As a useful heuristic, we use Gavish and
Donoho's estimator on the pooled data to estimate the shared subspace
dimension, $s$.

Given estimates of $s$ and $r$, we then proceed with Bayesian
inference of the eigenvectors, eigenvalues, and noise variance using
each of the three inferential models.  When applying the shared
subspace estimator, we use the EM algorithm presented in Section
\ref{sec:em} first, to estimate $VV^T$ and then use Bayesian inference
to estimate the posterior distribution of all remaining quantities
conditional on $V$.  For all covariance estimates, we compute Stein's
loss
$L_S( \Sigma_k , \hat\Sigma_k) = \text{tr}( \Sigma_k^{-1} \hat
\Sigma_k ) - \log |\Sigma_k^{-1} \Sigma_k | - p$
for each group.  Under Stein's loss, the Bayes estimator is the
inverse of the posterior mean of the precision matrix,
$\hat \Sigma_{k} = \Exp{ \Sigma_k^{-1} | \bl S_k}^{-1}$ which we
estimate using MCMC samples.  In Table \ref{table:groupLoss} we report
the average group loss
$L(\Sigma_1, ..., \Sigma_K; \hat\Sigma_1, ..., \hat\Sigma_K ) =
\frac{1}{K} \sum_k L_S( \Sigma_k , \hat\Sigma_k)$
and 95\% intervals for the estimates derived using each inferential
model.

Naturally, the estimates with the lowest loss are derived from the
inferential model that matches the data generating model. However, the
shared subspace estimator has small loss under model misspecification
relative to the alternative.  For example, when the data is generated
using the complete pooling model, the shared subspace estimator has
almost four times smaller loss than the no pooling estimator.  When
the data is generated under the no pooling model, the shared subspace
estimator is over an order of magnitude better than the complete
pooling estimator.  Thus, shared subspace estimation can be an
appropriate choice when the amount of similarity between groups is not
known a priori.  

The shared subspace model combined with the rank
estimation procedure suggested by \citet{Gavish2014} leads to an
estimator which adapts to the amount of similarity between groups.
Importantly, the shared subspace model can be viewed as a
generalization of the ``no pooling'' spiked covariance model.  The two
are equivalent when we let the shared subspace span the full space,
e.g. $V \in \mathcal{O}(p)$.  Figure \ref{fig:sdimension} depicts the
loss for data generated under the no pooling model and fit with shared
subspace estimators with different dimension.  As the shared subspace
dimension increases to the dimension of the feature space,
$s \rightarrow p$, the loss for the shared subspace estimator quickly
coverges to the loss for the individual spiked covariance estimator.
Note that span($[U_1, ..., U_K]) \leq rK$, thus even when there is
little similarity between the eigenvectors from each group, the shared
subspace estimator with $s = rK$ will perform comparably to the ``no
pooling'' estimator, provided that we can identify a subspace where
span($\hat{V})$ is close to $\text{span}([U_1, ..., U_K])$.

For any given subspace we can identify whether the space is
inappropriate for some groups, either because the dimension is too
small or the space itself does not represent every group equally.
Specifically, we estimate which groups are well represented on the
subspace and which groups are not using the proportion of variance
estimator from Section \ref{sec:em} (Equation \label{eqn:ratio}).
Figure \ref{fig:evalRatios} depicts the estimated proportion of
variance on the subspace for each group in two different simulated
datasets.  When the shared subspace model is appropriate, the subspace
spanned by $\hat{V}$ explains most of the variance in $\Sigma_k$, for
all groups.  When the estimated subspace is too small, the proportion
of variance will be small for at least some of the groups.

Ultimately, in many real world examples, we are more interested in
finding a useful space to interpret and compare differences between
covariance matrices than we are in finding the lowest-loss estimator
for the full covariance matrices.  In this context, the relevant
quantity is often an estimate of the projected covariance,
$V^T\Sigma V$.  In Figure \ref{fig:lossBoxplot} we depict boxplots for
the distribution of losses,
$L_S(\hat{V}^T\Sigma_k\hat{V}, \hat{V}^T\hat{\Sigma}_k\hat{V})$, on
simulated data.  In the left boxplot the data come from a
2-dimensional shared subspace model and on the right plot the data
come from a 10-dimensional shared subspace model.  For both boxplots,
estimates $\hat{V}^T\hat{\Sigma}_k\hat{V}$ are derived from assumed
2-dimensional model. This figure demonstrates that the loss for
estimating the projected covariance matrix is not significantly larger
when we underestimate the dimension of the true shared subspace.

\begin{table}
\begin{center}
  \begin{tabular}{ l  l | c | c | c |}
    \multicolumn{2}{c}{} & \multicolumn{3}{c}{\textbf{Inferential Model}} \\
  \multicolumn{2}{c|}{}  & Shared Subspace & Complete Pooling & No pooling \\  \cline{2-5}
    \multirow{3}{*}{\rotatebox[origin=c]{90}{\textbf{Data Model}}} & 
   Shared Subspace & 0.8 (0.7, 0.9) & 2.1 (1.7, 2.6) & 3.0 (2.9, 3.2) \\ \cline{2-5}
   & Complete Pooling & 0.8 (0.7, 0.9) & 0.7 (0.6, 0.8) & 3.0 (2.9, 3.2)\\ \cline{2-5}
   & No pooling & 7.1 (6.2, 8.0) & 138.2 (119, 153) & 3.0 (2.9, 3.2) \\ \cline{2-5}
  \end{tabular}
  \caption[Table caption text]{Stein's loss (and 95\% intervals) for
    different inferential and data generating models.  For each of
    $K=10$ groups, we simulate a $p=200$ dimensional spiked covariance
    matrix for rank $r=2$.  For data from a shared subspace model the
    eigenvectors of the covariance matrix share a subspace dimension
    $s=2$.  All results are based on $n_k = 50$ observations per
    group.  }
\label{table:groupLoss}
\end{center}
\end{table}


\begin{figure}[t]
    \centering
    \begin{subfigure}[b]{0.45\textwidth}
        \includegraphics[width=\textwidth]{Figs/LossVsDimension}
        \caption{Ratio}
        \label{fig:sdimension}
    \end{subfigure}
    ~ %add desired spacing between images, e. g. ~, \quad, \qquad, \hfill etc. 
      %(or a blank line to force the subfigure onto a new line)
    \begin{subfigure}[b]{0.45\textwidth}
        \includegraphics[width=\textwidth]{Figs/lossBoxplot}
        \caption{Posterior}
        \label{fig:lossBoxplot}
    \end{subfigure}
    \caption{Left) Stein's loss for data generated under the ``no
      pooling'' model and fit using the shared subspace model.  As the
      shared subspace dimension increases, $s \rightarrow p$, the loss
      converges to the loss from the independent spiked covariance
      estimates (dashed line). Right) Distribution of Stein's loss
      $L_S(\hat{V}^T\Sigma_k\hat{V}, \hat{V}^T\hat{\Sigma_k}\hat{V})$
      for data generated from a two dimensional shared subspace (left
      box) and data generated with a ten dimensional shared subspace
      (right).  In both cases the estimates were computed assuming a
      two-dimensional subspace, $s$ and matrix rank $r=2$.  Even when
      we underestimate the dimension of the shared subspace, the
      average loss of the estimator for the projected covariance matrices is
      similar to the loss under the correctly specified model. }
\end{figure}



\begin{figure}[!ht]
  \centering
    \includegraphics[width=0.75\textwidth]{Figs/evalRatios}
    \caption{Proportion of variance explained by the estimated
      subspace, $\hat{V}\hat{V}^T$, for each of 10 groups.  Left) Data
      generated and fit under the shared subspace with $s=r=2$.  The
      estimated two dimensional subspace explains nearly all of the
      variance in each of the ten groups.  Right) Data generated from
      the shared subspace model with $s=10$ and $r=2$ but estimated
      assuming $s=r=2$.  In this case, a two dimensional subspace does
      not capture the proportion of variance for all groups.  Here,
      the inferred subspace explains over 50\% of the variance for
      groups 2, 5, 7, 9 and 10 but almost none of the variance for
      groups 6 and 8. }
\label{fig:evalRatios}
\end{figure}



\section{Analysis of Biological Data}
\label{sec:bio}
As in many natural systems, high dimensional biological data is often
very structured and thus can be well understood on a relatively low
dimensional subspace. For example, with gene expression
data, the effective dimensionality is thought to scale with the number
of gene regulatory modules, not the number of genes themselves
\citep{Heimberg2016}.  As such, differences in gene expression across
groups should be expressed in terms of differences between these
regulatory modules rather than strict differences between expression
levels.  Such differences can be examined on a subspace that reflects
the correlations resulting from these modules.

With this in mind, we apply the shared subspace spiked covariance model to data in two
different biological examples.  In the first, we compare gene
expression data from juveniles with different subtypes of leukemia.  Even
after removing mean effects, the data indicate significant differences
in covariance matrices between groups.  In the second example, we
compare covariance matrices in a metabolomic analysis of fly data.  In
this example, after controlling for mean effects due to fly age, gender and
genotype, we find little evidence of differences in metabolite
covariances between groups. 

In these analyses we employ a variation of the shared
subspace model.  Specifically, we define the $s$ by $s$ matrix $\psi_k$ as

\begin{equation}
\psi_k =\left( \begin{array}{cc}
O_k\Lambda_kO_k^T & 0  \\
0 & D  \end{array} \right)
\end{equation}

For the following analyses $O_k\Lambda_kO_k^T$ is a rank $r=2$ matrix
and $D$ is an $(s-r)$-dimensional diagonal matrix.  We write
$V = [V_1\, V_2]$, with $V_1 \in \mathcal{V}_{p,r}$ as the basis for a
r-dimensional shared subspace that explains the differences between
groups and $V_2 \in \mathcal{V}_{p,(s-r)}$ corresponding to the remaining $(s-r)$
eigenvectors common to all groups.  Importantly, we make no assumptions about the
magntidue of the eigenvalues $\Lambda_k$ relative to the eigenvalues
$D$, so that the largest eigenvectors of $\psi_k$ may be in $V_2$.  

We find this formulation useful becasue in real world analyses,
differences between groups may not manifest themselves in the first
principle components.  For instance, in genetic analyses of humans,
there may be large person to person variability common to all groups
which is unrelated to the grouping of those subjects based on the
phenotype of interest.  Differences between groups may manifest themselves
as changes in the smaller principle components. This formulation
allows, complete pooling of most eigenvectors, yet a space to identify
and compare relevant differences.  Moreover, visualizing and
interpreting differences between groups on higher dimensional spaces
is a challenging problem.  Using this framework we can easily understand the
largest differences across groups in a two dimensional space while
explaining the remaining variability with a anisoptropic shared
structure.

\paragraph{Analysis of Gene Expression Data}

Here, we demonstrate the utility of the shared subspace
covariance estimator for exploring differences in the covariability of
gene expression levels in young adults with different subtypes of
pediatric acute lymphoblastic leukemia \citep{Yeoh2002}.  The raw data
consists of gene expression levels for over 12,000 genes in seven
different subtypes of leukemia: BCR-ABL, E2A-PBX1, Hyperdip50, MLL,
T-ALL, TEL-AML1 and a seventh group for unidentified subtypes.  The
groups have corresponding sample sizes of
$n = (15, 27, 64, 20, 43, 79, 79)$.  Although there are over 12,000
genes, the vast majority of expression levels are missing.
Thus, we restrict our attention to the genes for which less than half
of the values are missing and use Amelia, a software package for
missing value imputation, to fill in the remaining missing values
\citep{Amelia}.  After restricting the data in this way, $p=3124$
genes remain.  Prior to analysis, we demean both the rows and columns of
the gene expression levels in each group.

In Figure \ref{fig:leukemia} we plot the results of our analysis.
Panel \ref{fig:leukemiaRatio} shows that over $40\%$ of the variance
can be explained by the estimated two-dimensional subspace for all groups, with 4 of
the 6 groups over 60\%.  Panel \ref{fig:leukemiaPosterior} reflects some significant
differences in the posterior distribution of eigenvalues and
eigenvectors between groups.  The $x$-axis corresponds to the
orientation of the principle axis of $\psi_k$ on $V$ and the $y$-axis
corresponds to the ratio of the larger eigenvalue over the smaller
eigenvalue.  Several groups show significantly different orientations
for the first principle component of the subspace $\hat{V}$.  Further,
the subtype E2A-PBX1 has a larger eigenvalue ratio which reflects a
more correlated distribution on the subspace spanned by $V$.  The
TEL-AML1 and Hyperdip50 appear to cluster together which suggests
there is little detectable difference between their covariance
structure.  Note that when the eigenvalue ratio is close to one (as is
the case for the ``other'' subtype), the distribution of expression
values on the subspace is nearly spherical and thus the orientation of
the primary eigenvector is at best weakly identifiable.  This is
reflected in the wide posterior range of orientations of the first
principle component.

Further intuition about the differences in covariances between groups
can be understood in the biplot in Figure \ref{fig:leukemiaBiplot}.
Here, we plot the contours of the covariance matrices for three of
subtypes of Leukemia.  We also plot different shapes for the 1\% of genes with
the largest loadings on the first two columns of $V$ and the
remaining values as light grey dots.  The genes with the largest
loadings are clustered by quadrant and listed in the corresponding
table.  Even though we remove all mean-level differences in gene
expressions between groups, we are still able to identify genes with
known connections to cancer and leukemia.  As one example, MYC (top
right) is an oncogene with well established association to many
cancers \citep{Dang2012}.

\begin{figure}[t]
    \centering
    \begin{subfigure}[b]{0.45\textwidth}
        \includegraphics[width=\textwidth]{Figs/leukemiaRatio-04-27}
        \caption{Proportion of Variance on $\hat{V}$}
        \label{fig:leukemiaRatio}
    \end{subfigure}
    ~ %add desired spacing between images, e. g. ~, \quad, \qquad, \hfill etc. 
      %(or a blank line to force the subfigure onto a new line)
    \begin{subfigure}[b]{0.45\textwidth}
        \includegraphics[width=\textwidth]{Figs/leukemiaPosterior-04-27}
        \caption{95\% Posterior Regions }
        \label{fig:leukemiaPosterior}
    \end{subfigure}
\caption{}
\label{fig:leukemia}
\end{figure}

In this figure, genes that fall in the upper right quadrant have
positive loadings in both columns of $V$. Genes in the upper left
corner have positive loadings on the first column of $V$ but negative
loadings on the second column of $V$.  The principle axis for a group
aligns in the direction of genes that exhibit the largest variability
in that group.  Genes which lie in a direction orthogonal to the
principle axis exhibit reduced variability, relative to the other
groups. For instance, genes in the upper left (triangles) exhibit
large, positively correlated variability in the Hyperdip50 group.  In
this same group, there is reduced varaibility, relative to the other
groups, among the genes in the upper right of the plot (squares),
since this cluster of genes lie in a direction orthogonal to the
principle axis.  In contrast, the BCR-ABL group is aligned primarily
with the second column of $V$, which means that the genes indicated by
squares and circles vary significantly in this group.  Genes in the
square group are anti-correlated with those in the circle group, since
the their loadings have opposite signs on $V2$.


  \begin{figure}[!ht]
    \centering
    \includegraphics[width=0.5\textwidth]{Figs/leukemia-biplot-05-13}
    \qquad
\raisebox{1.25\height}{
\footnotesize
\input{Figs/biplot.tab}
}
\caption{Left) Variant of a biplot with contours for three leukemia
  subtypes and the loadings for each gene on the first two columns of
  $\hat{V}$.  All gene's loadings are displayed in light gray, and the
  top 1\% of genes with with the largest magnitude loadings are
  displayed as either a triangle, square or circle depending on which
  quadrant they lie in.  Right) List of the gene's with the largest
  loadings, grouped by quadrant. }
\label{fig:leukemiaBiplot}
  \end{figure}

\paragraph{Metabolomic Analysis}

Next, we apply shared subspace group covariance estimation to a
metabolomic data analysis on fly aging \citep{Hoffman2014}.  We bin
the data to include groups of flies less than 10 days (young), between
10 and 51 days (middle) and greater than 51 days (old), and further
split by gender. The sample sizes range from 43 to 59 flies per group.
We analyze metabolomic data corresponding to metabolites with 3714
mass-charge ratios.  After removing mean effects due to age and gender
we fit the shared subspace estimator and compare covariances between
groups.  Interestingly, we identify a subspace that explains almost
all of the variability in the first few components of $\Sigma_k$ for
all groups (Figure \ref{fig:dmelanRatio}). However, we see little
evidence for differences in metabolite covariances on this subspace,
as evidenced by the large overlap in posterior distributions for each group
(Figure \ref{fig:dmelanPosterior}). This is indicative of large
amount of variation which is common across all groups for the first
$s$ principle components. Differences in covariation across may
manifest themselves for the smaller principle components, but are too
small to detect given the sample sizes.

\begin{figure}[t]
    \centering
    \begin{subfigure}[b]{0.45\textwidth}
        \includegraphics[width=\textwidth]{Figs/dmelanRatio-04-28}
        \caption{Ratio}
        \label{fig:dmelanRatio}
    \end{subfigure}
    ~ %add desired spacing between images, e. g. ~, \quad, \qquad, \hfill etc. 
      %(or a blank line to force the subfigure onto a new line)
    \begin{subfigure}[b]{0.45\textwidth}r
        \includegraphics[width=\textwidth]{Figs/dmelanPosterior-04-28}
        \caption{Posterior}
        \label{fig:dmelanPosterior}
    \end{subfigure}
    \caption{a) Proportion of variance on $\hat{V}$ is close to one
      for all groups suggesting that blah. However, there is
      significant overlap in the posterior distribution of $\Lambda_k$
      and $U_k$. This suggests that there is little difference in the
      covariance matrices between groups, on this particular subspace.  }
\end{figure}

\section{Discussion}

In this paper, we present a method for comparing differences between
covariance matrices on a low dimensional subspace.  We provide an EM
algorithm for estimating the subspace which explains the largest
proportion of variance .

Variation that differs across groups versus variation which is the
same across groups.

Importantly, our approach can be combined with existing hierarchical
models for multi-group covariance estimation.  For instance, we can
incorporate additional shrinkage as in \citet{Hoff2009} by utilizing
non-uniform Bingham prior distributions for the subspace eigenvectors.
Alternatively, we can summarize the estimated group covariance
matrices by thresholding entries of the precision matrix to visualize
differences between groups using a graphical model \citep{}.

Full Bayesian approach using Riemmanian manifold sampler?

\bibliographystyle{plainnat}
\bibliography{refs.bib}

\section{Appendix}

\end{document}

